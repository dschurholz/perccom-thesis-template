\section{INTRODUCTION}\label{sec:intro}

\subsection[BACKGROUND]{Background}
\par The goal of this report is to give instructions how to write a master's thesis. The report is only an outline, a model which will be extended according to the contents of the actual work for the thesis. For more information, read ``Final thesis instructions'' in the study guide, discuss with your supervisor, visit appropriate course web pages, see for example [1].  When you are referring to pages in the (WWW)\nomenclature[WWWW]{WWW}{World Wide Web}, see the instruction for citing and referencing from the LUT Library web site. Do not use footnotes or do not write URLs within the text. 

\par Introduction contains three subsections: background, goals and delimitations, and description of the structure of the thesis. Use this sectioning.  Subsection 1.1 includes an introduction to the background for the work.  Remember that the abstract is a separate piece of text. The introduction should be written independently such that one does not need to read the abstract to understand the introduction.  The introduction is written in a general level instead of many details present. These details will be explained later starting from section 2. The paragraphs have more than only one sentence. In the thesis, this subsection occupies from 1 to 2 pages.

\par Remember to introduce the abbreviations when they are used in the text for the first time. For example: ``This thesis is about the games played in National Hockey League (NHL)\nomenclature[WNHL]{NHL}{National Hockey League} in seasons 1900-2000. The annual penalties in NHL have ...''.

\par The introduction is written such that the reader is interested to continue to read the full thesis. And if this interest is arisen then the author is ready to give some general descriptions for the contents of the thesis and reading guidelines for the rest of the text in the thesis.

\clearpage

\subsection[GOALS AND DELIMITATIONS]{Goals and delimitations}

\par Express the goals for the work, include also the delimitations. This way the reader knows when the results are valid and she can place the work in a proper framework and scope. It is also important to say, what is not done during the work for the thesis. Then the thesis will show how the goals are met. In the thesis this subsection occupies from 1 to 2 pages.

\subsection[STRUCTURE OF THE THESIS]{Structure of the thesis}

\par This subsection contains a short description for the contents of the thesis. The contents of each section are characterized with one or two sentences. For example: ``Section 2 contains a description of the ...''. In the thesis this subsection occupies at most one page, in many cases half a page is enough. At this point, one should thoroughly consider the structure of work. Discuss with your supervisor about the structure.
\citep{Dur00}.
